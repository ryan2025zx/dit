\documentclass{article}
\usepackage[margin=64pt]{geometry} % 控制全局格式信息
\usepackage{graphicx} % 引入图片信息
\usepackage{xcolor} % 引入颜色控制
\usepackage{amsmath} % 引入额外的数学功能
\usepackage{amssymb} % 引入额外的数学符号
\usepackage{listings} % 插入代码
\usepackage{float} % 取消浮动
\usepackage{enumitem} % 个性化enumerate的项
\usepackage{pifont} % 可以输入定制化符号
\usepackage[utf8]{inputenc} % 引入table of contents
\usepackage[backend=biber,style=numeric]{biblatex}  % 使用biblatex来实现参考文献的引用
\addbibresource{content/references.bib}  % 使用 .bib 文件
\usepackage{pxfonts} % lstlisting 环境关键字加粗

\setlist[enumerate]{itemsep=0pt, parsep=0pt} 
% \setcounter{secnumdepth}{0} % 取消section序号
\setlength{\parindent}{0pt} % 取消全文首行缩进

\newcommand{\ztitle}[1]{
    \begin{center}
        {\fontsize{28pt}{36pt}\selectfont\textbf{#1}}
    \end{center}
}

\newcommand{\zinfo}[3]{
    \vspace{24pt}
    \colorbox{gray!30}{
        \parbox{\dimexpr\linewidth-38pt}{
            \begin{tabular}{@{}l@{\hspace{24pt}}l@{}}
                \textbf{Student Name:} & #1 \\
                \textbf{Student ID:} & #2 \\
                \textbf{Student Email:} & #3 \\
            \end{tabular}
        }
    }
    \vspace{24pt}
}

% \lstset{
%     language=Python,
%     backgroundcolor=\color{black},
%     basicstyle=\color{green}\ttfamily,
%     commentstyle=\color{yellow},
%     morecomment=[l]{\#}, % 这个设置是用来额外指定或修改注释的识别方式。morecomment 字段允许你定义更多的注释样式。在这个例子中,[l]{\#} 表示以 # 字符开始的行为单行注释。[l] 代表“line comment”,即该符号后面到行末的文本都被视为注释。
%     breaklines=true,
% }

\renewcommand{\lstlistingname}{Code}
\lstset{
    language=Python,         % 设置语言
    basicstyle=\ttfamily,     % 设置基本字体样式为等宽字体
    commentstyle=\itshape,    % 设置注释为斜体
    keywordstyle=\bfseries,   % 设置关键词为粗体
    breaklines=true,          % 自动换行
    frame=single,             % 添加边框
    numbers=left,             % 在左侧显示行号
    numberstyle=\tiny,        % 行号的字体大小
    tabsize=4,                % 设置Tab为4个空格
    showstringspaces=false,   % 不显示字符串中的空格
}

% \lstset{
%     backgroundcolor=\color{gray!10}, % 设置代码背景颜色
%     basicstyle=\ttfamily\small,      % 设置代码的基本样式
%     breaklines=true,                 % 代码过长时自动换行
%     captionpos=b,                    % 标题位置(b表示底部)
%     keepspaces=true,                 % 保持空格不被忽略
%     numbers=left,                    % 行号显示在左侧
%     numberstyle=\tiny\color{gray},   % 行号的样式
%     showstringspaces=false,          % 字符串中显示空格
%     frame=single,                    % 代码周围显示框架
%     language=python                  % 默认语言设置为bash
% }

% 此处开始,设置cmark和xmark
\newlength{\dingwidth}
\settowidth{\dingwidth}{\ding{55}} % 初始化宽度,通常选取最宽的一个

\newcommand{\uniformding}[1]{\makebox[\dingwidth][c]{#1}} % 创建等宽盒子命令

\newcommand{\cmark}{\uniformding{\ding{51}}} % 使用 \uniformding 使 \ding{51} 等宽
\newcommand{\xmark}{\uniformding{\ding{55}}} % 使用 \uniformding 使 \ding{55} 等宽
% 此处结束,设置cmark和xmark